\documentclass{article}

\title{Text analysis}
\author{Adam Martinez}
\date{}

\begin{document}

\maketitle

\section*{Things to take note of}

\begin{itemize}
    \item Read the data at the end of text on the author and source.
    \item Take note of a single idea per paragraph.
\end{itemize}

\section*{Topic}

\begin{itemize}
    \item Use a brief phrase with an abstract noun.
    \item A one-liner.
    \item Question 3 (writing) can provide good insight.
\end{itemize}

\section*{Basic parts}

\begin{itemize}
    \item Narrative texts
    \begin{itemize}
        \item Approach
        \item Node
        \item Ending
    \end{itemize}
    \item Expositive texts
    \begin{itemize}
        \item Introduction
        \item Expositive body
        \item Conclusion
    \end{itemize}
    \item Argumentative texts
    \begin{itemize}
        \item Introduction
        \item Argumentative body
        \item Conclusion
    \end{itemize}
\end{itemize}

\subsection*{Common examples of each}

\begin{itemize}
    \item Narrative texts
    \begin{itemize}
        \item Short stories
        \item Novels
        \item Tales
    \end{itemize}
    \item Expositive texts
    \begin{itemize}
        \item Textbooks
        \item Manuals
        \item Encyclopedias
    \end{itemize}
    \item Argumentative texts
    \begin{itemize}
        \item Essays
        \item Articles
        \item Editorials
    \end{itemize}
\end{itemize}

\subsection*{Thesis}

It is only present in argumentative texts.

\begin{itemize}
    \item Sintetising
    \begin{itemize}
        \item Appears in the conclusion.
    \end{itemize}
    \item Analysing
    \begin{itemize}
        \item Appears in the introduction.
    \end{itemize}
    \item Framed
    \begin{itemize}
        \item Appears in the introduction and conclusion.
    \end{itemize}
\end{itemize}

\section*{Summary}

\begin{itemize}
    \item 3rd person.
    \item Present tense.
    \item Use notes on each paragraph taken during reading.
\end{itemize}

\section*{Textual typology}

\subsection*{Narrative texts}

\begin{itemize}
    \item They have characters to which a series of events happen.
    \item They follow a specific structure.
    \item Time can be linear or not.
    \item Past simple, periphrastic and imperfect forms; present simple.
    \item Settings can be real or imaginary.
\end{itemize}

\subsection*{Argumentative texts}

\begin{itemize}
    \item They follow a specific structure and have a thesis.
    \item They have a series of arguments that support the thesis.
    \item They are subjective, and thus have a ton of modal vocab.
    \item They use either the 1st person singular or plural.
    \item They use some literary devices.
\end{itemize}

\subsection*{Expositive texts}

\begin{itemize}
    \item They follow a specific structure.
    \item Enunciation and the referential function of the language prevail.
    \item Denoation is abundant; abstract nouns are common.
    \item Clear and simple sentences are used.
    \item They use the 3rd person singular or plural.
\end{itemize}

\section*{Registres}

\subsection*{Formal}

\begin{itemize}
    \item Ideas are perfectly structured in paragraphs.
    \item Subordination is abundant, and long sentences are common.
\end{itemize}

\subsubsection*{Scientific}

\begin{itemize}
    \item Technical vocabulary is abundant.
    \item Neologisms and words related to the field are common.
\end{itemize}

\subsubsection*{Literary}

\begin{itemize}
    \item Literary devices are common.
    \item Other registres can be used to recreate a specific atmosphere.
\end{itemize}

\subsection*{Standard}

\begin{itemize}
    \item Uses a neutral vocabulary.
    \item Ideas are well-structured.
    \item Grammar and spelling are correct.
\end{itemize}

\subsection*{Informal}

\begin{itemize}
    \item Errors in grammar and spelling can be present.
    \item Idioms are common.
    \item Broad terms are used.
\end{itemize}

\subsection*{Vulgar}

\begin{itemize}
    \item Errors in grammar and spelling are very common.
    \item Ordre is not present.
    \item Slang, imprecise terms and vulgarisms are common.
\end{itemize}

\section*{Typographical elements}

\subsection*{Bold}

\begin{itemize}
    \item Titles.
    \item Highlight important ideas.
\end{itemize}

\subsection*{Underline}

\begin{itemize}
    \item Highlight important ideas.
\end{itemize}

\subsection*{Italics}

\begin{itemize}
    \item Foreign words.
    \item Quotes.
    \item Convey irony.
    \item Titles of works.
    \item Registre change.
\end{itemize}

\subsection*{Quotes}

\begin{itemize}
    \item Direct speech.
    \item Foreign words.
    \item Registre change.
    \item Convey irony.
\end{itemize}

\subsection*{Hyphen - Dash}

\begin{itemize}
    \item Interventions in a dialogue.
    \item Explanations or examples.
    \item Ideas in an outline.
\end{itemize}

\subsection*{Parenthesis}

\begin{itemize}
    \item Explanations or examples.
\end{itemize}

\section*{Literary devices}

\subsection*{Alliteration}

Repetition of the same sounds.

\subsection*{Onomatopoeia}

Words that imitate sounds.

\subsection*{Paranomasia}

Wordplay with similar-sounding words.
Use of a word in different senses.

\subsection*{Asyndeton}

Omission of conjunctions.

\subsection*{Ellipsis}

Omission of words.

\subsection*{Enumeration}

Listing of elements.

\subsection*{Hyperbaton}

Change in the order of words.

\subsection*{Parallelism}

Repetition of the same structure.

\subsection*{Polysindeton}

Repetition of conjunctions.

\subsection*{Antithesis}

Opposition of ideas.

\subsection*{Irony}

Saying the opposite of what is meant in a sarcastic way.

\subsection*{Repetition}

Repetition of words or structures.

\subsection*{Comparison}

Comparison of two elements.

\subsection*{Metaphor}

Implicit comparison of two seemingly unrelated elements.

\subsection*{Metonymy}

Substitution of a word for another related word.

\subsection*{Personification}

Attribution of human qualities to inanimate objects.

\subsection*{Synaesthesia}

Mixing of senses.

\section*{Voices of speech}

Apart from the narrator, there is an enunciator and a speaker.
The first is anyone mentioned in quotes, and the second is one to which the text
is explictely dedicated.

\subsection*{Narrator}

\subsubsection*{Narrator depending on the POV}

\begin{itemize}
    \item External, 3rd person.
    \item Internal:
    \begin{itemize}
        \item Protagonist, 1st person.
        \item Witness, 3rd and 1st person.
        \item Secondary.
    \end{itemize}
\end{itemize}

\subsubsection*{Narrator depending on the knowledge}

\begin{itemize}
    \item Omniscient.
    \item Identified, unique perspective.
    \item Objectivist, multiple perspectives.
\end{itemize}

\subsection*{1st person enunciator roles}

There is a modest and inclusive plural; pretty self-explanatory.

\subsection*{Other enunciators}

If there is any change to the narrator throughout the text, it is important to
mention so.

\begin{itemize}
    \item Direct speech, dialogues or quotes.
    \item Indirect speech, diction verbs and conjunctions.
    \item Free indirect speech, no diction verbs but the direct form is kept.
    \item Internal monologue, thoughts of the character.
    \item Intertextual references, quotes or allusions.
\end{itemize}

\section*{Modalisation}

\begin{itemize}
    \item 
\end{itemize}

\end{document}