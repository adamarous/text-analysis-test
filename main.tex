\documentclass{article}

\usepackage{amssymb}

\title{Text analysis}
\author{Adam Martinez}
\date{}

\begin{document}

\maketitle

\section*{Things to take note of}

\begin{itemize}
    \item Read the data at the end on the author and source.
    \item Take note of a single idea per paragraph.
\end{itemize}

\section*{Topic}

\begin{itemize}
    \item Use a one-liner phrase with an abstract noun.
    \item Question 3 (writing) can provide good insight.
\end{itemize}

\section*{Basic parts}

\begin{itemize}
    \item Narrative texts
    \begin{itemize}
        \item Approach
        \item Node
        \item Ending
    \end{itemize}
    \item Expositive texts
    \begin{itemize}
        \item Introduction
        \item Expositive body
        \item Conclusion
    \end{itemize}
    \item Argumentative texts
    \begin{itemize}
        \item Introduction
        \item Argumentative body
        \item Conclusion
    \end{itemize}
\end{itemize}

\subsection*{Thesis}

It is only present in argumentative texts.

\begin{itemize}
    \item Sintetising
    \begin{itemize}
        \item Appears in the conclusion.
    \end{itemize}
    \item Analysing
    \begin{itemize}
        \item Appears in the introduction.
    \end{itemize}
    \item Framed
    \begin{itemize}
        \item Appears in the introduction and conclusion.
    \end{itemize}
\end{itemize}

\section*{Summary}

\begin{itemize}
    \item 3rd person.
    \item Present tense.
    \item Use notes on each paragraph taken during reading.
\end{itemize}

\section*{Textual typology}

\subsection*{Narrative texts}

\begin{itemize}
    \item They have characters that go through a series of events.
    \item They follow a specific structure.
    \item The timeline can be linear or non-linear.
    \item Past simple, periphrastic and imperfect forms; present simple.
    \item Settings can be real or imaginary.
\end{itemize}

\subsection*{Argumentative texts}

\begin{itemize}
    \item They follow a specific structure and have a thesis.
    \item They have arguments that support the thesis.
    \item They are subjective, and thus have a ton of modal vocabulary.
    \item They use either the 1st person singular or plural.
    \item They use some literary devices.
\end{itemize}

\subsection*{Expositive texts}

\begin{itemize}
    \item They follow a specific structure.
    \item Enunciation and the referential function of the language prevail.
    \item Denotation is abundant; abstract nouns are common.
    \item Clear and simple sentences are used.
    \item They use the 3rd person singular or plural.
\end{itemize}

\section*{Registres}

\subsection*{Formal}

\begin{itemize}
    \item Ideas are perfectly structured in paragraphs.
    \item Subordination is abundant, and long sentences are common.
\end{itemize}

\subsubsection*{Scientific}

\begin{itemize}
    \item Technical vocabulary is abundant.
    \item Neologisms and words related to the field are common.
\end{itemize}

\subsubsection*{Literary}

\begin{itemize}
    \item Literary devices are common.
    \item Other registres can be used to recreate a specific atmosphere.
\end{itemize}

\subsection*{Standard}

\begin{itemize}
    \item Uses a neutral vocabulary.
    \item Ideas are well-structured.
    \item Grammar and spelling are correct.
\end{itemize}

\subsection*{Informal}

\begin{itemize}
    \item Errors in grammar and spelling can be present.
    \item Idioms are common.
    \item Broad words are used.
\end{itemize}

\subsection*{Vulgar}

\begin{itemize}
    \item Errors in grammar and spelling are very common.
    \item Ordre is not present.
    \item Imprecise words, slang and vulgarisms are common.
\end{itemize}

\section*{Typographical elements}

\subsection*{Bold}

\begin{itemize}
    \item Titles.
    \item Highlight important ideas.
\end{itemize}

\subsection*{Underline}

\begin{itemize}
    \item Highlight important ideas.
\end{itemize}

\subsection*{Italics}

\begin{itemize}
    \item Foreign words.
    \item Citations.
    \item Ironies.
    \item Titles of works.
    \item Registre changes.
\end{itemize}

\subsection*{Quotes}

\begin{itemize}
    \item Foreign words.
    \item Ironies.
    \item Registre changes.
    \item Direct speech.
\end{itemize}

\subsection*{Hyphen - Dash}

\begin{itemize}
    \item Interventions in a dialogue.
    \item Explanations or examples.
    \item Ideas in an outline.
\end{itemize}

\subsection*{Parenthesis}

\begin{itemize}
    \item Explanations or examples.
\end{itemize}

\section*{Literary devices}

\subsection*{Alliteration}

Repetition of the same sounds.

\subsection*{Onomatopoeia}

Words that imitate sounds.

\subsection*{Paranomasia}

Wordplay with similar-sounding words.
Use of a word in different senses.

\subsection*{Asyndeton}

Omission of conjunctions.

\subsection*{Ellipsis}

Omission of words.

\subsection*{Enumeration}

Listing of elements.

\subsection*{Hyperbaton}

Change in the order of words.

\subsection*{Parallelism}

Repetition of the same structure.

\subsection*{Polysindeton}

Repetition of conjunctions.

\subsection*{Antithesis}

Opposition of ideas.

\subsection*{Irony}

Saying the opposite of what is meant in a sarcastic way.

\subsection*{Repetition}

Repetition of words or structures.

\subsection*{Comparison}

Comparison of two elements.

\subsection*{Metaphor}

Implicit comparison of two seemingly unrelated elements.

\subsection*{Metonymy}

Substitution of a word for another related word.

\subsection*{Personification}

Attribution of human qualities to inanimate objects.

\subsection*{Synaesthesia}

Mixing of senses.

\section*{Voices of speech}

Apart from the narrator, there is an enunciator and a speaker.
The first refers to anyone mentioned in citations, and the second is the one to
which the text might be explictely dedicated.

\subsection*{Narrator}

\subsubsection*{Narrator depending on its POV}

\begin{itemize}
    \item External; 3rd person.
    \item Internal:
    \begin{itemize}
        \item Protagonist; 1st person.
        \item Witness; 3rd and 1st person.
        \item Secondary.
    \end{itemize}
\end{itemize}

\subsubsection*{Narrator depending on its knowledge}

\begin{itemize}
    \item Omniscient.
    \item Identified; unique perspective.
    \item Objectivist; multiple perspectives.
\end{itemize}

\subsection*{1st person enunciator roles}

There is a modest and an inclusive plural; pretty self-explanatory.

\subsection*{Other enunciators}

If there are any changes to the narrator throughout the text, it is important to
comment on them.

\begin{itemize}
    \item Direct speech; dialogues or citations.
    \item Indirect speech; diction verbs and conjunctions.
    \item Free indirect speech; no diction verbs but the direct form is kept.
    \item Internal monologue; thoughts of the character.
    \item Intertextual references; citations or allusions.
\end{itemize}

\section*{Modalisation}

\begin{itemize}
    \item 1st person singular or plural.
    \item Presence of a thesis.
    \item Use of arguments.
    \item Assessment vocabulary; verbs, nouns and adjectives.
    \item Obligation verbs.
\end{itemize}

\section*{Impersonalisation}

\begin{itemize}
    \item Impersonal verbs.
    \item Sentences with a subject of broad nature.
    \item 3rd person plural without a specific subject.
    \item 2nd person singular with an implicit \emph{me} subject.
    \item Infinitive verbs.
\end{itemize}

\section*{Dialectal varieties - Valencian/Catalan}

\begin{itemize}
    \item Singular possesive $\rightarrow$ \emph{-u-} or \emph{-v-}.
    \item 1st person singular indicative $\rightarrow$ \emph{-e -o} or
    \emph{$\varnothing$ -i -o -u}.
    \item Subjunctive and imperative $\rightarrow$ \emph{-a -en} or
    \emph{-i -in}.
    \item Present simple indicative, subjunctive and imperative $\rightarrow$
    \emph{-ix -isca} or \emph{-eix -esca}.
    \item Proper nouns $\rightarrow$ without article or with an article before
    the noun.
    \item Negation $\rightarrow$ \emph{no} or \emph{no pas}.
\end{itemize}

\section*{Phonetics}

\subsection*{Deaf sounds}

The trick is to identify the sound in either \emph{petaca}, \emph{feixos} or
\emph{cotxe}; though if a vowel follows, it is surely sonorous.
If possible, try to pronounce it yourself and see whether your vocal cords
vibrate or not.

Another formula is to see whether there's a sonorous sound on one word and a
deaf one on the next; if so, the sound is deaf.

\subsection*{Open sounds}

\subsubsection*{With an \emph{e}}

\begin{itemize}
    \item In front of a syllable containing \emph{i} or \emph{u}.
    \item In front of \emph{l}, \emph{r} or \emph{rr}.
    \item In words stressed on the third-to-last syllable.
    \item In scientific words.
    \item In educated terminations; \emph{-ecte/a}, \emph{-epte/a}.
    \item In words ending in a \emph{-eu} diphthong.
    \item If the Spanish word ends in a \emph{-ie} diphthong.
\end{itemize}

\subsubsection*{With an \emph{o}}

\begin{itemize}
    \item Directly in front of or in front of a syllable containing \emph{i} or
    \emph{u}.
    \item In words stressed on the third-to-last syllable.
    \item In the \emph{o} of decreasing diphthongs.
    \item In scientific words.
    \item The \emph{pero} preposition and \emph{allo, aixo, aco}.
    \item In words ending in \emph{-nos}, \emph{-osa}, \emph{-oc/a}, \emph{-oc},
    \emph{-of/a}, \emph{-oig}, \emph{-oja}, \emph{-ol/a}, \emph{-olt/a},
    \emph{-ossa}, \emph{-ost/a}, \emph{-ot/a}.
    \item In words with a stressed \emph{o} followed by a consonantal group
    containing \emph{l} or \emph{r}.
    \item If the Spanish word ends in a \emph{-ue} diphthong.
\end{itemize}

\section*{Weak pronouns}

\subsection*{Substitutions}

\subsubsection*{CD}

\begin{itemize}
    \item Introduced by \emph{aixo}, \emph{allo} or a subordinate sentence
    $\rightarrow$ \emph{ho}.
    \item Introduced by articles, possessives or demonstratives $\rightarrow$
    \emph{el la els les}.
    \item Undefined $\rightarrow$ \emph{en}.
\end{itemize}

\subsubsection*{Subject}

\begin{itemize}
    \item \emph{en}.
\end{itemize}

\subsubsection*{CI}

\begin{itemize}
    \item \emph{li, els}.
\end{itemize}

\subsubsection*{CRV, CCL, CCM}

\begin{itemize}
    \item Introduced by \emph{de} $\rightarrow$ \emph{en}.
    \item Introduced by any other preposition $\rightarrow$ \emph{hi}.
\end{itemize}

\subsubsection*{CP}

\begin{itemize}
    \item \emph{hi}.
\end{itemize}

\subsubsection*{CA}

\begin{itemize}
    \item Defined $\rightarrow$ \emph{el la els les}.
    \item Undefined; speaks of the subject $\rightarrow$ \emph{ho}.
\end{itemize}

\subsubsection*{CN}

\begin{itemize}
    \item \emph{en}.
\end{itemize}

\subsection*{Things to take note of}

\begin{itemize}
    \item The apostrophe should be as further to the right as possible.
    \item The ordre of pronouns should be: \emph{se CI CD en hi}.
    \item The \emph{la} pronoun doesn't apostrophise before \emph{hi}.
    \item Careful not to use \emph{se}, \emph{li} is the correct pronoun in most
    cases.
\end{itemize}

\end{document}